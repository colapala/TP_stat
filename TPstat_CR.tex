\PassOptionsToPackage{unicode=true}{hyperref} % options for packages loaded elsewhere
\PassOptionsToPackage{hyphens}{url}
%
\documentclass[]{article}
\usepackage{lmodern}
\usepackage{amssymb,amsmath}
\usepackage{ifxetex,ifluatex}
\usepackage{fixltx2e} % provides \textsubscript
\ifnum 0\ifxetex 1\fi\ifluatex 1\fi=0 % if pdftex
  \usepackage[T1]{fontenc}
  \usepackage[utf8]{inputenc}
  \usepackage{textcomp} % provides euro and other symbols
\else % if luatex or xelatex
  \usepackage{unicode-math}
  \defaultfontfeatures{Ligatures=TeX,Scale=MatchLowercase}
\fi
% use upquote if available, for straight quotes in verbatim environments
\IfFileExists{upquote.sty}{\usepackage{upquote}}{}
% use microtype if available
\IfFileExists{microtype.sty}{%
\usepackage[]{microtype}
\UseMicrotypeSet[protrusion]{basicmath} % disable protrusion for tt fonts
}{}
\IfFileExists{parskip.sty}{%
\usepackage{parskip}
}{% else
\setlength{\parindent}{0pt}
\setlength{\parskip}{6pt plus 2pt minus 1pt}
}
\usepackage{hyperref}
\hypersetup{
            pdftitle={TP Statistique},
            pdfauthor={Cédric Milinaire, Corentin Laharotte},
            pdfborder={0 0 0},
            breaklinks=true}
\urlstyle{same}  % don't use monospace font for urls
\usepackage[margin=1in]{geometry}
\usepackage{color}
\usepackage{fancyvrb}
\newcommand{\VerbBar}{|}
\newcommand{\VERB}{\Verb[commandchars=\\\{\}]}
\DefineVerbatimEnvironment{Highlighting}{Verbatim}{commandchars=\\\{\}}
% Add ',fontsize=\small' for more characters per line
\usepackage{framed}
\definecolor{shadecolor}{RGB}{248,248,248}
\newenvironment{Shaded}{\begin{snugshade}}{\end{snugshade}}
\newcommand{\AlertTok}[1]{\textcolor[rgb]{0.94,0.16,0.16}{#1}}
\newcommand{\AnnotationTok}[1]{\textcolor[rgb]{0.56,0.35,0.01}{\textbf{\textit{#1}}}}
\newcommand{\AttributeTok}[1]{\textcolor[rgb]{0.77,0.63,0.00}{#1}}
\newcommand{\BaseNTok}[1]{\textcolor[rgb]{0.00,0.00,0.81}{#1}}
\newcommand{\BuiltInTok}[1]{#1}
\newcommand{\CharTok}[1]{\textcolor[rgb]{0.31,0.60,0.02}{#1}}
\newcommand{\CommentTok}[1]{\textcolor[rgb]{0.56,0.35,0.01}{\textit{#1}}}
\newcommand{\CommentVarTok}[1]{\textcolor[rgb]{0.56,0.35,0.01}{\textbf{\textit{#1}}}}
\newcommand{\ConstantTok}[1]{\textcolor[rgb]{0.00,0.00,0.00}{#1}}
\newcommand{\ControlFlowTok}[1]{\textcolor[rgb]{0.13,0.29,0.53}{\textbf{#1}}}
\newcommand{\DataTypeTok}[1]{\textcolor[rgb]{0.13,0.29,0.53}{#1}}
\newcommand{\DecValTok}[1]{\textcolor[rgb]{0.00,0.00,0.81}{#1}}
\newcommand{\DocumentationTok}[1]{\textcolor[rgb]{0.56,0.35,0.01}{\textbf{\textit{#1}}}}
\newcommand{\ErrorTok}[1]{\textcolor[rgb]{0.64,0.00,0.00}{\textbf{#1}}}
\newcommand{\ExtensionTok}[1]{#1}
\newcommand{\FloatTok}[1]{\textcolor[rgb]{0.00,0.00,0.81}{#1}}
\newcommand{\FunctionTok}[1]{\textcolor[rgb]{0.00,0.00,0.00}{#1}}
\newcommand{\ImportTok}[1]{#1}
\newcommand{\InformationTok}[1]{\textcolor[rgb]{0.56,0.35,0.01}{\textbf{\textit{#1}}}}
\newcommand{\KeywordTok}[1]{\textcolor[rgb]{0.13,0.29,0.53}{\textbf{#1}}}
\newcommand{\NormalTok}[1]{#1}
\newcommand{\OperatorTok}[1]{\textcolor[rgb]{0.81,0.36,0.00}{\textbf{#1}}}
\newcommand{\OtherTok}[1]{\textcolor[rgb]{0.56,0.35,0.01}{#1}}
\newcommand{\PreprocessorTok}[1]{\textcolor[rgb]{0.56,0.35,0.01}{\textit{#1}}}
\newcommand{\RegionMarkerTok}[1]{#1}
\newcommand{\SpecialCharTok}[1]{\textcolor[rgb]{0.00,0.00,0.00}{#1}}
\newcommand{\SpecialStringTok}[1]{\textcolor[rgb]{0.31,0.60,0.02}{#1}}
\newcommand{\StringTok}[1]{\textcolor[rgb]{0.31,0.60,0.02}{#1}}
\newcommand{\VariableTok}[1]{\textcolor[rgb]{0.00,0.00,0.00}{#1}}
\newcommand{\VerbatimStringTok}[1]{\textcolor[rgb]{0.31,0.60,0.02}{#1}}
\newcommand{\WarningTok}[1]{\textcolor[rgb]{0.56,0.35,0.01}{\textbf{\textit{#1}}}}
\usepackage{graphicx,grffile}
\makeatletter
\def\maxwidth{\ifdim\Gin@nat@width>\linewidth\linewidth\else\Gin@nat@width\fi}
\def\maxheight{\ifdim\Gin@nat@height>\textheight\textheight\else\Gin@nat@height\fi}
\makeatother
% Scale images if necessary, so that they will not overflow the page
% margins by default, and it is still possible to overwrite the defaults
% using explicit options in \includegraphics[width, height, ...]{}
\setkeys{Gin}{width=\maxwidth,height=\maxheight,keepaspectratio}
\setlength{\emergencystretch}{3em}  % prevent overfull lines
\providecommand{\tightlist}{%
  \setlength{\itemsep}{0pt}\setlength{\parskip}{0pt}}
\setcounter{secnumdepth}{0}
% Redefines (sub)paragraphs to behave more like sections
\ifx\paragraph\undefined\else
\let\oldparagraph\paragraph
\renewcommand{\paragraph}[1]{\oldparagraph{#1}\mbox{}}
\fi
\ifx\subparagraph\undefined\else
\let\oldsubparagraph\subparagraph
\renewcommand{\subparagraph}[1]{\oldsubparagraph{#1}\mbox{}}
\fi

% set default figure placement to htbp
\makeatletter
\def\fps@figure{htbp}
\makeatother


\title{TP Statistique}
\author{Cédric Milinaire, Corentin Laharotte}
\date{4 avril 2020}

\begin{document}
\maketitle

Voici le plan de ce qui sera fait dans le TP.

\hypertarget{visualisation-de-chemins}{%
\section{0. Visualisation de chemins}\label{visualisation-de-chemins}}

Lecture du fichier des villes :

\begin{Shaded}
\begin{Highlighting}[]
\NormalTok{villes <-}\StringTok{ }\KeywordTok{read.csv}\NormalTok{(}\StringTok{'DonneesGPSvilles.csv'}\NormalTok{,}\DataTypeTok{header=}\OtherTok{TRUE}\NormalTok{,}\DataTypeTok{dec=}\StringTok{'.'}\NormalTok{,}\DataTypeTok{sep=}\StringTok{';'}\NormalTok{,}\DataTypeTok{quote=}\StringTok{"}\CharTok{\textbackslash{}"}\StringTok{"}\NormalTok{)}
\KeywordTok{str}\NormalTok{(villes)}
\end{Highlighting}
\end{Shaded}

\begin{verbatim}
## 'data.frame':    22 obs. of  5 variables:
##  $ EU_circo : Factor w/ 7 levels "Centre","Est",..: 6 6 4 2 7 4 2 1 2 4 ...
##  $ region   : Factor w/ 22 levels "Alsace","Aquitaine",..: 22 9 19 10 2 4 8 3 5 17 ...
##  $ ville    : Factor w/ 22 levels "Ajaccio","Amiens",..: 11 1 2 3 4 5 6 7 8 9 ...
##  $ latitude : num  45.7 41.9 49.9 47.2 44.8 ...
##  $ longitude: num  4.847 8.733 2.3 6.033 -0.567 ...
\end{verbatim}

Représentation des chemins par plus proches voisins et du chemin optimal
:

\begin{Shaded}
\begin{Highlighting}[]
\NormalTok{coord <-}\StringTok{ }\KeywordTok{cbind}\NormalTok{(villes}\OperatorTok{$}\NormalTok{longitude,villes}\OperatorTok{$}\NormalTok{latitude)}
\NormalTok{dist <-}\StringTok{ }\KeywordTok{distanceGPS}\NormalTok{(coord)}
\NormalTok{voisins <-}\StringTok{ }\KeywordTok{TSPnearest}\NormalTok{(dist)}

\NormalTok{pathOpt <-}\StringTok{ }\KeywordTok{c}\NormalTok{(}\DecValTok{1}\NormalTok{,}\DecValTok{8}\NormalTok{,}\DecValTok{9}\NormalTok{,}\DecValTok{4}\NormalTok{,}\DecValTok{21}\NormalTok{,}\DecValTok{13}\NormalTok{,}\DecValTok{7}\NormalTok{,}\DecValTok{10}\NormalTok{,}\DecValTok{3}\NormalTok{,}\DecValTok{17}\NormalTok{,}\DecValTok{16}\NormalTok{,}\DecValTok{20}\NormalTok{,}\DecValTok{6}\NormalTok{,}\DecValTok{19}\NormalTok{,}\DecValTok{15}\NormalTok{,}\DecValTok{18}\NormalTok{,}\DecValTok{11}\NormalTok{,}\DecValTok{5}\NormalTok{,}\DecValTok{22}\NormalTok{,}\DecValTok{14}\NormalTok{,}\DecValTok{12}\NormalTok{,}\DecValTok{2}\NormalTok{)}

\KeywordTok{par}\NormalTok{(}\DataTypeTok{mfrow=}\KeywordTok{c}\NormalTok{(}\DecValTok{1}\NormalTok{,}\DecValTok{2}\NormalTok{),}\DataTypeTok{mar=}\KeywordTok{c}\NormalTok{(}\DecValTok{1}\NormalTok{,}\DecValTok{1}\NormalTok{,}\DecValTok{2}\NormalTok{,}\DecValTok{1}\NormalTok{))}
\KeywordTok{plotTrace}\NormalTok{(coord[voisins}\OperatorTok{$}\NormalTok{chemin,], }\DataTypeTok{title=}\StringTok{'Plus proches voisins'}\NormalTok{)}
\KeywordTok{plotTrace}\NormalTok{(coord[pathOpt,], }\DataTypeTok{title=}\StringTok{'Chemin optimal'}\NormalTok{)}
\end{Highlighting}
\end{Shaded}

\includegraphics{TPstat_CR_files/figure-latex/unnamed-chunk-2-1.pdf}

Les longueurs des trajets (à vol d'oiseau) valent respectivement, pour
la méthode des plus proches voisins :

\begin{verbatim}
## [1] 4303.568
\end{verbatim}

et pour la méthode optimale :

\begin{verbatim}
## [1] 3793.06
\end{verbatim}

Ceci illustre bien l'intérêt d'un algorithme de voyageur de commerce.
Nous allons dans la suite étudier les performances de cet algorithme.

\hypertarget{comparaison-dalgorithmes}{%
\section{1. Comparaison d'algorithmes}\label{comparaison-dalgorithmes}}

Nombre de sommets fixes et graphes ``identiques''.

\begin{Shaded}
\begin{Highlighting}[]
\NormalTok{      n <-}\StringTok{ }\DecValTok{10}
\NormalTok{sommets <-}\StringTok{ }\KeywordTok{data.frame}\NormalTok{(}\DataTypeTok{x =} \KeywordTok{runif}\NormalTok{(n), }\DataTypeTok{y =} \KeywordTok{runif}\NormalTok{(n))}
\NormalTok{  couts <-}\StringTok{ }\KeywordTok{distance}\NormalTok{(sommets)}
\end{Highlighting}
\end{Shaded}

\hypertarget{longueur-des-chemins}{%
\subsection{1.1. Longueur des chemins}\label{longueur-des-chemins}}

Comparaison des longueurs de différentes méthodes :

\begin{itemize}
\tightlist
\item
  boxplots
\end{itemize}

\includegraphics{TPstat_CR_files/figure-latex/unnamed-chunk-6-1.pdf}

\begin{itemize}
\tightlist
\item
  test entre `nearest' et `branch'
\end{itemize}

On souhaite comparer les méthodes des plus proches voisins et
Branch\&Bound. On réalise le test sur l'espérance : (H0) m\_nn - m\_b
\textless{}= 0 \textless{}=\textgreater{} m\_nn\textless{}=m\_b (H1)
m\_nn - m\_b \textgreater{} 0 \textless{}=\textgreater{}
m\_nn\textgreater{}m\_b

Région critique à 5\% : \{T \textgreater{} t\_n-1;2a\}

T =

\begin{verbatim}
## [1] 5.799712
\end{verbatim}

t\_a =

\begin{verbatim}
## [1] 2.009575
\end{verbatim}

\begin{verbatim}
## [1] "T > t_a"
## [1] "On peut rejeter H0"
\end{verbatim}

On peut rejeter H0, et affirmer avec un risque de 5\% que les chemins
des plus proches voisins sont en moyenne plus long que ceux de
Branch\&Bound.

(Pour celui là je suis pas sûr que ce soit cette méthode qu'il faille
utiliser !)

Vérification

\begin{Shaded}
\begin{Highlighting}[]
\NormalTok{t <-}\StringTok{ }\KeywordTok{t.test}\NormalTok{(cheminNearest,}\DataTypeTok{mu=}\KeywordTok{mean}\NormalTok{(cheminBranch) ,}\DataTypeTok{alternative =} \StringTok{"greater"}\NormalTok{) }\CommentTok{#alternative = "greater" pour H1 m > m0}
\KeywordTok{print}\NormalTok{(t)}
\end{Highlighting}
\end{Shaded}

\begin{verbatim}
## 
##  One Sample t-test
## 
## data:  cheminNearest
## t = 5.7997, df = 49, p-value = 2.378e-07
## alternative hypothesis: true mean is greater than 2.850538
## 95 percent confidence interval:
##  3.053389      Inf
## sample estimates:
## mean of x 
##  3.135872
\end{verbatim}

On peut aussi faire t.test(cheminNearest, cheminBranch,alternative =
``greater'')

\begin{itemize}
\tightlist
\item
  tests 2 à 2
\end{itemize}

Ici (H0) mi=mj (H1) mi!=mj

\begin{verbatim}
## 
##  Pairwise comparisons using t tests with pooled SD 
## 
## data:  results and methods 
## 
##                   branch  nearest nearest_insertion repetitive_nn
## nearest           0.00078 -       -                 -            
## nearest_insertion 0.02272 0.94921 -                 -            
## repetitive_nn     0.94921 0.01702 0.20157           -            
## two_opt           0.09341 0.53849 0.94921           0.53849      
## 
## P value adjustment method: holm
\end{verbatim}

A COMMENTER

Si on accepte de se tromper de alpha=5\%, on rejette H0 si la pvaleur de
(i,j) est inférieure à alpha/10 (vu dans l'énoncé donc 0.05). ? pas sûr
de ça !

\hypertarget{temps-de-calcul}{%
\subsection{1.2. Temps de calcul}\label{temps-de-calcul}}

Comparaison des temps à l'aide du package microbenchmark.

Application de microbenchmark :

\begin{verbatim}
## Unit: microseconds
##                                  expr      min        lq      mean    median
##      TSPsolve(couts, "repetitive_nn") 3457.635 3769.9185 3923.0684 3962.6845
##  TSPsolve(couts, "nearest_insertion")  562.313  586.6955  697.4603  689.1540
##            TSPsolve(couts, "two_opt")  319.913  341.6255  390.4273  367.6880
##            TSPsolve(couts, "nearest")    8.698    9.7350   14.6087   13.1775
##             TSPsolve(couts, "branch")  972.159 1855.8935 3051.9718 2684.9120
##         uq       max neval
##  4021.5930  4365.725    20
##   749.2350   975.256    20
##   414.2715   636.984    20
##    14.8550    45.089    20
##  3318.3975 10146.461    20
\end{verbatim}

\hypertarget{etude-de-la-complexituxe9-de-lalgorithme-branch-and-bound}{%
\section{2. Etude de la complexité de l'algorithme Branch and
Bound}\label{etude-de-la-complexituxe9-de-lalgorithme-branch-and-bound}}

\hypertarget{comportement-par-rapport-au-nombre-de-sommets-premier-moduxe8le}{%
\subsection{2.1. Comportement par rapport au nombre de sommets : premier
modèle}\label{comportement-par-rapport-au-nombre-de-sommets-premier-moduxe8le}}

Récupération du temps sur 10 graphes pour différentes valeurs de \(n\).

\includegraphics{TPstat_CR_files/figure-latex/unnamed-chunk-16-1.pdf}
Les nombres représentés sont les numéros de colonnes de la valeur à la
nième ligne !

Ajustement du modèle linéaire de \(\log(temps)^2\) en fonction de \(n\).

\begin{verbatim}
## 
## Call:
## lm(formula = vect_temps ~ vect_dim)
## 
## Residuals:
##     Min      1Q  Median      3Q     Max 
## -71.369 -19.114   3.286  21.137  60.481 
## 
## Coefficients:
##             Estimate Std. Error t value Pr(>|t|)    
## (Intercept)  58.9734     5.4253   10.87   <2e-16 ***
## vect_dim     14.1713     0.4186   33.86   <2e-16 ***
## ---
## Signif. codes:  0 '***' 0.001 '**' 0.01 '*' 0.05 '.' 0.1 ' ' 1
## 
## Residual standard error: 26.74 on 168 degrees of freedom
## Multiple R-squared:  0.8722, Adjusted R-squared:  0.8714 
## F-statistic:  1146 on 1 and 168 DF,  p-value: < 2.2e-16
\end{verbatim}

COMMENTER

Analyse de la validité du modèle :

\begin{itemize}
\tightlist
\item
  pertinence des coefficients et du modèle
\end{itemize}

\includegraphics{TPstat_CR_files/figure-latex/unnamed-chunk-18-1.pdf}

COMMENTER

\begin{itemize}
\tightlist
\item
  étude des hypothèses sur les résidus.
\end{itemize}

PAS SUR QUE LE PARAMETRE SOIT temps.lm

(H0) les résidus suivent une loi normale (H1) les résidus ne suivent pas
une loi normale

On prend un risque alpha=5\%

\begin{verbatim}
## 
##  Shapiro-Wilk normality test
## 
## data:  residuals(temps.lm)
## W = 0.97119, p-value = 0.001325
\end{verbatim}

\begin{verbatim}
## [1] "p-valeur < alpha"
## [1] "On peut rejeter H0"
\end{verbatim}

On rejette H0, donc nous pouvons affirmer que les résidus ne suivent pas
une loi normale.

\hypertarget{comportement-par-rapport-au-nombre-de-sommets-uxe9tude-du-comportement-moyen}{%
\subsection{2.2. Comportement par rapport au nombre de sommets : étude
du comportement
moyen}\label{comportement-par-rapport-au-nombre-de-sommets-uxe9tude-du-comportement-moyen}}

Récupération du temps moyen.

Ajustement du modèle linéaire de \(\log(temps.moy)^2\) en fonction de
\(n\).

\begin{verbatim}
## 
## Call:
## lm(formula = vect_temps_moy ~ vect_dim_moy)
## 
## Residuals:
##     Min      1Q  Median      3Q     Max 
## -36.915 -13.217   8.921  13.477  25.991 
## 
## Coefficients:
##              Estimate Std. Error t value Pr(>|t|)    
## (Intercept)   59.0607    12.8854   4.584 0.000358 ***
## vect_dim_moy  14.5681     0.9941  14.654  2.7e-10 ***
## ---
## Signif. codes:  0 '***' 0.001 '**' 0.01 '*' 0.05 '.' 0.1 ' ' 1
## 
## Residual standard error: 20.08 on 15 degrees of freedom
## Multiple R-squared:  0.9347, Adjusted R-squared:  0.9304 
## F-statistic: 214.7 on 1 and 15 DF,  p-value: 2.697e-10
\end{verbatim}

Analyse de la validité du modèle :

\begin{itemize}
\tightlist
\item
  pertinence des coefficients et du modèle
\end{itemize}

\includegraphics{TPstat_CR_files/figure-latex/unnamed-chunk-23-1.pdf}

\begin{itemize}
\tightlist
\item
  étude des hypothèses sur les résidus.
\end{itemize}

PAS SUR QUE LE PARAMETRE SOIT temps.lm\_moy

(H0) les résidus suivent une loi normale (H1) les résidus ne suivent pas
une loi normale

On prend un risque alpha=5\%

\begin{verbatim}
## 
##  Shapiro-Wilk normality test
## 
## data:  residuals(temps.lm_moy)
## W = 0.88931, p-value = 0.04513
\end{verbatim}

\begin{verbatim}
## [1] "p-valeur < alpha"
## [1] "On peut rejeter H0"
\end{verbatim}

On ne peut pas rejeter H0. Donc on peut assurer avec un risque de 5\%
que les résidus suivent une loi normale. Validité du modèle ???

\hypertarget{comportement-par-rapport-uxe0-la-structure-du-graphe}{%
\subsection{2.3. Comportement par rapport à la structure du
graphe}\label{comportement-par-rapport-uxe0-la-structure-du-graphe}}

Lecture du fichier `DonneesTSP.csv'.

\begin{Shaded}
\begin{Highlighting}[]
\NormalTok{data.graph <-}\StringTok{ }\KeywordTok{read.csv}\NormalTok{(}\StringTok{'DonneesTSP.csv'}\NormalTok{,}\DataTypeTok{header=}\OtherTok{TRUE}\NormalTok{,}\DataTypeTok{dec=}\StringTok{'.'}\NormalTok{,}\DataTypeTok{sep=}\StringTok{';'}\NormalTok{,}\DataTypeTok{quote=}\StringTok{"}\CharTok{\textbackslash{}"}\StringTok{"}\NormalTok{)}
\KeywordTok{str}\NormalTok{(data.graph)}
\end{Highlighting}
\end{Shaded}

\begin{verbatim}
## 'data.frame':    70 obs. of  1 variable:
##  $ tps.dim.mean.long.mean.dist.sd.dist.mean.deg.sd.deg.diameter: Factor w/ 70 levels "10727263,12,0.197262188031624,0.37067582512226,0.267885367956632,11,0,1",..: 49 10 70 24 55 5 26 39 48 35 ...
\end{verbatim}

Ajustement du modèle linéaire de \(\log(temps.moy)^2\) en fonction de
toutes les variables présentes. Modèle sans constante.

\begin{Shaded}
\begin{Highlighting}[]
\CommentTok{#lm(log(data.graph$tps)~., data = data.graph)}
\end{Highlighting}
\end{Shaded}

Mise en \oe uvre d'une sélection de variables pour ne garder que les
variables pertinentes.

Analyse de la validité du modèle :

\begin{itemize}
\item
  pertinence des coefficients et du modèle,
\item
  étude des hypothèses sur les résidus.
\end{itemize}

\end{document}
